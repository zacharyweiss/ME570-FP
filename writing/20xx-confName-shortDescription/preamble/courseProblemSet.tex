%make underscore work without escaping in text mode
\usepackage{underscore}

%Common style that will be used for titles
\newcommand{\titlestyle}{\color{DodgerBlue3}\sffamily}

%Perfect 1-inch margins
\usepackage[margin=1in]{geometry}
\usepackage[medium,compact]{titlesec}
\usepackage{longtable}
\titleformat*{\section}{\titlestyle \Large}
\titleformat*{\subsection}{\titlestyle \large \itshape}

%Lists (requires enumitem package):
%Font for description
\setlist[description]{font=\color{DodgerBlue3}}
%Compact lists
\setlist[1]{\labelindent=\parindent}
\setlist[1]{topsep=3pt, itemsep=0pt}

%Format main title
\let\prevtitle\title
\renewcommand{\title}[1]{\prevtitle{\titlestyle #1}}
\let\prevauthor\author
\renewcommand{\author}[1]{\prevauthor{\titlestyle #1}}
\let\prevdate\date
\renewcommand{\date}[1]{\prevdate{\titlestyle #1}}

%Displayed quotations (used to define an environment for function specification)
\usepackage{csquotes}


%Include code
\usepackage{listings}

%formatting for the description of a function's input/output arguments
\mdfdefinestyle{functionspecification}{%
linecolor=DodgerBlue3,
rightline=false, topline=false, bottomline=false,
outerlinewidth=1.35,
roundcorner=10pt,
innerleftmargin=1.2em,
innertopmargin=0.3em,
innerbottommargin=0.3em,
leftmargin=1pt,
skipabove=0.5em,  
skipbelow=0.5em,  
backgroundcolor=white}

\newcommand{\dstar}{$\bigstar$}


%Facilities for markers file
% open file the beginning of the document
\AtBeginDocument{%
  \newwrite\markerstream
  \immediate\openout\markerstream=\jobname.mlg
}
% close the file at the end of the document
\AtEndDocument{%
  \immediate\closeout\markerstream
}
%definition similar to \typeout, but for markers
\makeatletter
\newcommand{\markerout}[1]{\begingroup \set@display@protect \immediate \write \markerstream {#1}\endgroup}
\makeatother

\usetikzlibrary{intersections}

\titleformat{\section}{\titlestyle\large\bfseries}{Problem \thesection:}{0.5ex}{}
%\titleformat{\subsection}{\normalfont\bfseries}{Problem \thesubsection:}{0.5ex}{}
\titleformat{\paragraph}[runin]{\titlestyle\normalsize\bfseries}{}{2pt}{}

%Macro for paragraph header for preparation steps to questions
%\newcommand{\preparation}{\smallskip\par\noindent\textbf{\titlestyle Preparation.~}}

\newcommand{\preparation}[1][]{%
\smallskip\par\noindent%
\textbf{\titlestyle Preparation \displayifnotempty{~(}{#1}{)}.~}}

%New custom counter for question
\newcounter{question}[section]

%Macro for writing points for the current question to the log file.
\newcommand{\pointmarker}{\typeout{Question points marker [\thesection.\thequestion] [\questionpoints]}}

\makeatletter
%Makes questions that:
%- are automatically numbered using the format <sectionNumber>.<questionNumber>
%- can be referred to using the usual \label-\ref mechanism
%- if the optional argument is not empty, this is displayed in parenthesis (typically, this will be a \points{n} or a \optional command). If the optional argument is empty, a single point is silently added to the point counter.
\newcommand{\question}[1][]{%
\smallskip\par\noindent%
\refstepcounter{question}\zerovar{\questionpoints}%
\sbox\boxifnotempty{\textbf{\titlestyle #1}}% for some reason, if the box is saved inside the \textbf{\titlestyle ...} (to avoid repeating the style), the box counts as empty for \ifempty, so I need to repeat the style here.
\textbf{\titlestyle Question \thesection.\thequestion\displayifnotempty{~(}{\usebox{\boxifnotempty}}{)}.~}%
\ifempty{\usebox{\boxifnotempty}}{\increasepoints{1}}%
\def\@currentlabel{{\thesection.\thequestion}}\pointmarker}
\makeatother
\newcommand{\hintquestion}[1]{\paragraph{Hint
    for question \ref{#1}:~}}
\newcommand{\hint}{\paragraph{Hint:~}}

\newcommand{\answer}[1][]{\smallskip\leavevmode\\\noindent\textbf{Answer \thesection.\thequestion#1:~}}

\newcommand{\updated}[1]{{\color{red} #1}}

\newcolorlabel{labeloptional}{DodgerBlue3}
\newcommand{\optional}{\labeloptional{optional}}
\newcolorlabel{labelprovided}{Orange1}
\newcommand{\provided}{\labelprovided{provided}}
\newcolorlabel{labelvideo}{OliveDrab3}
\newcommand{\video}{\labelvideo{video}}
\newcolorlabel{labelimage}{DarkOrchid2}
\newcommand{\image}{\labelimage{image}}
\newcolorlabel{labelcode}{SlateBlue2}
\newcommand{\code}{\labelcode{code}}
\newcolorlabel{labelscan}{Chocolate3}
\newcommand{\scan}{\labelcode{scan}}
\newcolorlabel{labelnew}{OliveDrab3}
\newcommand{\lbnew}{\labelnew{new}}

%\newcommand{\optional}{{\titlestyle optional}}

\tikzset{point/.style={inner sep=0pt,fill=black,circle,minimum size=4pt}}

%macros for keeping track of points (total, and for current question)
\zerovar{\totalpoints}
\zerovar{\bonuspoints}
\zerovar{\questionpoints}
\newcommand{\increasepoints}[1]{\addtovar{\totalpoints}{#1}\addtovar{\questionpoints}{#1}}
\newcommand{\points}[1]{#1 point\ifthenelse{\equal{#1}{1}}{}{s}\increasepoints{#1}}
\makeatletter
\newcommand{\totalpointswithbonus}[1]{\pgfmathparse{\storeddef{totalpoints}+#1}\pgfmathresult\typeout{Total points marker [\pgfmathresult]}}
\makeatother

\newstoreddef{totalpoints} %remember total number of points in aux file

% answer boxes
\newcommand{\vspaceboxdelta}{3mm}
\newcommand{\vspacebox}[1]{\par\bigskip{\tikz{\useasboundingbox (0,-\vspaceboxdelta) rectangle (\textwidth,#1\baselineskip); \draw[gray!20!white,line width=3pt] (-\vspaceboxdelta,-\vspaceboxdelta) rectangle (\textwidth+2*\vspaceboxdelta,#1\baselineskip+\vspaceboxdelta);}}\par}
\newcommand{\vspaceoneline}{\vspacebox{2}}
\newcommand{\vspacetwolines}{\vspacebox{3}}
\newcommand{\vspacethreelines}{\vspacebox{4}}
\newcommand{\vspacefourlines}{\vspacebox{5}}

% text fields
\newcommand{\textrowfield}[1]{\par\noindent#1: \enspace\hrulefill}

% formatting for the description of a function's input/output arguments
\usepackage[framemethod=tikz]{mdframed}
\mdfdefinestyle{functionspecification}{%
linecolor=DodgerBlue3,
rightline=false, topline=false, bottomline=false,
outerlinewidth=1.35,
roundcorner=10pt,
innerleftmargin=1.2em,
innertopmargin=0.3em,
innerbottommargin=0.3em,
leftmargin=1pt,
skipabove=0.5em,  
skipbelow=0.5em,  
backgroundcolor=white}
\mdfdefinestyle{classspecification}{%
linecolor=DarkOliveGreen3,
rightline=false, topline=false, bottomline=false,
outerlinewidth=1.35,
roundcorner=10pt,
innerleftmargin=1.2em,
innertopmargin=0.3em,
innerbottommargin=0.3em,
leftmargin=1pt,
skipabove=0.5em,  
skipbelow=0.5em,  
backgroundcolor=white}
%Note: I was not able to avoid duplication of options.

\newenvironment{functionspecification}{\markerout{Marker function [start]}\begin{mdframed}[style=functionspecification]}
{\markerout{Marker function [end]}\end{mdframed}}%
\newenvironment{classspecification}{\markerout{Marker class [start]}\begin{mdframed}[style=classspecification]}
{\markerout{Marker class [end]}\end{mdframed}}%
%\NewEnviron{fdescription}{\par\noindent{\titlestyle Description}: \BODY\markerout{Marker text [description] [start]^^J\BODY^^JMarker text [description] [end]}\par}
\NewEnviron{fdescription}{\par\noindent{\titlestyle Description}: \BODY\markerout{Marker text [description] [start]^^J\BODY^^JMarker text [description] [end]}\par}
\newenvironment{cdescription}{\fdescription}{\endfdescription}
\NewEnviron{frequirements}{\par\noindent{\titlestyle Requirements}: \BODY\markerout{Marker text [requirements] [start]^^J\BODY^^JMarker text [requirements] [end]}\par}
\newcommand{\cname}[1]{\noindent{\titlestyle Class name}: \var{#1}\par}
\newenvironment{finputs}{\markerout{Marker argument_group [start] [inputs]}\noindent{\titlestyle Input arguments}\begin{itemize}}{\end{itemize}\markerout{Marker argument_group [end]}}
\newenvironment{foutputs}[1][Output]{\markerout{Marker argument_group [start] [outputs]}\noindent{\titlestyle #1 arguments}\begin{itemize}}{\end{itemize}\markerout{Marker argument_group [end]}}
\newcommand{\freturns}[2][]{\noindent{\titlestyle Returns}: \var{#2}\displayifnotempty{ (type }{\var{#1}}{)}}
\newcommand{\argument}[4][]{\markerout{Marker argument [start]}\markerout{Marker argument [name] [#2]}\var{#2} %
  \ifnotempty{#1#3}{(\ifnotempty{#3}{\markerout{Marker argument [dimensions] [#3] [#4]}dim. \vardim{#3}{#4}\ifnotempty{#1}{, }}%
  \displayifnotempty{\markerout{Marker argument [type] [#1]}type }{\var{#1}}{})}\markerout{Marker argument [end]}}
\newcommand{\argumentscalar}[2][]{\argument[#1]{#2}{}{}}
\newcommand{\fheader}[3][]{\noindent{\titlestyle Header}: \function[#1]{#2}{#3}\par}
\newcommand{\fname}[2][Function]{\noindent{\titlestyle #1 name}: \var{#2}\par}
\newcommand{\ffilename}[1]{\noindent{\titlestyle File name}: \file{#1}\par}
\newcommand{\functionheader}[3][]{\markerout{Marker function_header [#1] [#2] [#3]}\function[#1]{#2}{#3}\par}

%ROS node specification
\newenvironment{nodespecification}{\begin{mdframed}[style=functionspecification]%
\newcommand{\filename}[1]{\par{\titlestyle File name:} \file{##1}}
\renewenvironment{description}{\par\noindent{\titlestyle Description:}}{\par}%
\newenvironment{subscribers}{\par\noindent{\titlestyle Subscribes to topics:}}{}%
\newenvironment{publishers}{\par\noindent{\titlestyle Publishes to topics:}}{}%
%
}{\end{mdframed}}

\newcommand{\rostopic}[2]{\var{#1} (type \var{#2})}

% Macros for questions where student need to draw on axes
\newcommand{\drawingaxes}[1][]{\draw[gray] (-5,-5) grid (5,5);
  \draw[very thick, -latex] (-5.7,0) -- (5.7,0);
  \draw[very thick, -latex] (0,-5.7) -- (0,5.7); #1}
\newcommand{\shifteddrawingaxes}[1][]{
  \begin{scope}[xshift=15cm]
    \drawingaxes[#1]
  \end{scope}
}

\newcommand{\drawexercise}[2][]{\begin{center}
\begin{tikzpicture}[scale=0.6]
  \node[text width=5cm,anchor=west]{#2};
  \shifteddrawingaxes[#1]
\end{tikzpicture}
\end{center}
}
